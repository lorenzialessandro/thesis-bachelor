\chapter*{Sommario} % senza numerazione
\label{sommario}

\addcontentsline{toc}{chapter}{Sommario} % da aggiungere comunque all'indice
In questo capitolo introduttivo si descrivono il progetto di tirocinio, l'azienda e l'ambiente in cui è stato svolto, il contesto e lo scopo dell'elaborato, nonché le motivazioni e gli obiettivi prefissati. In conclusione si riporta la struttura del documento.

\section*{Il progetto}
\label{sec:progetto}
\addcontentsline{toc}{section}{Il progetto} % da aggiungere comunque all'indice
Il progetto di tirocinio ha avuto come scopo principale lo sviluppo di un gestionale completo per l'area formazione dell'azienda. È stato difatti realizzato un software per poter facilitare e strutturare tutti i processi di quella parte dell'organizzazione che si occupa di impostare, offrire e controllare tutti gli aspetti relativi ai corsi di formazione, dalla gestione completa dei percorsi offerti ai certificati rilasciati e a tutte le anagrafiche con cui quest’area interagisce.\\
\newline
Per questo lavoro è stata dedicata una prima parte all’analisi completa dei requisiti, seguendo i vari stadi dell'analisi del software \cite{analisis}, e successivamente è iniziata la progettazione dell’intero sistema, dal database alla parte di programmazione. Infine il sistema è stato configurato e reso disponibile agli utenti finali online.\\
\newline
Lo scopo di questo elaborato è quindi riportare e analizzare il contesto e il flusso di progettazione di questo applicativo in ogni sua fase.


\section*{Azienda e contesto}
\label{sec:contensto}
\addcontentsline{toc}{section}{Azienda e contesto} % da aggiungere comunque all'indice
L'esperienza di tirocinio è stata svolta presso Dream S.R.L. \cite{dream}, impresa fondata nel 2004 che opera nei settori della formazione e della consulenza aziendale con particolare attenzione alla trasformazione digitale, rivolta a piccole e medie imprese for profit e non profit. L'azienda offre ai clienti servizi di consulenza e sviluppo legati all’analisi dei fabbisogni, alla progettazione soluzioni ad hoc, al project management, alla gestione aziendale e altro. Inoltre, sulla parte formativa, promuove la crescita, l’apprendimento e il rafforzamento delle competenze individuali e di gruppo attraverso la progettazione, la gestione e la realizzazione di percorsi formativi, fatti su misura per le aziende.

\begin{figure}[h]
\centering
\includegraphics[scale=1.2]{img/DREAM_sito_logo_colori.png}
\caption{Logo Dream S.R.L.}
\label{fig:logodream}
\end{figure}
\noindent
È un'azienda di piccole - medie dimensioni con sedi a Trento, Tione di Trento e Verona.
Il tirocinio è stato svolto principalmente nella sede operativa di Trento in via lungadige Giacomo Leopardi, per un totale di 225 ore.


\section*{Obiettivi e motivazioni}
\label{sec:obiettivi}
\addcontentsline{toc}{section}{Obiettivi e motivazioni} % da aggiungere comunque all'indice
L'obiettivo principale dell'esperienza è stato lo sviluppo di strumenti per le richieste interne ed esterne all’azienda con attenzione ai progetti di
trasformazione digitale. In particolare buona parte del lavoro svolto è stata dedicata all'implementazione del gestionale per l’area aziendale interna legata alla formazione, come sopra descritto. Attraverso questo elaborato si descrive e si contestualizza, sia ad alto livello, che in modo dettagliato, il ciclo di crescita e sviluppo di questo applicativo.\\
\newline
Oltre a questo aspetto, durante il tirocinio, altre attività minori svolte sono state:
\begin{itemize}
    \item Supporto tecnico quotidiano all’area formazione e all’area consulenza, legato soprattutto alla gestione e monitoraggio della rete aziendale, del server, delle VPN e dei vari profili e permessi. Considerando che l’azienda lavora su sedi differenti e interagisce con un unico server nella sede di Trento si è intervenuti spesso nella gestione dei sistemi, con corrispondente risoluzione dei problemi. 
    \item Ricerca, scouting e implementazione di soluzioni adatte a rispondere alle esigenze tecniche dei clienti su diversi progetti legati alla digitalizzazione e all'integrazione.
\end{itemize}

\section*{Struttura dell'elaborato}
\label{sec:struttura}
\addcontentsline{toc}{section}{Struttura dell'elaborato} % da aggiungere comunque all'indice
Il documento è diviso nei seguenti capitoli:
\begin{enumerate}
    \item Ambito dell’elaborato e del progetto: si introducono alcuni concetti relativi ai Sistemi Informativi, collocando e definendo l'ambito del lavoro e del documento.
    \item Analisi dei requisiti: questo capitolo riporta l'analisi fatta nella prima parte del progetto. In particolare si descrivono e si raffigurano con diagrammi i requisiti funzionali e non funzionali, le attività, il contesto e i componenti.   
    \item Progettazione del database: dopo una parte introduttiva si illustra il processo di progettazione della base di dati utilizzata, con relativo diagramma EER e approfondimento sulla tecnica di normalizzazione.
    \item Sviluppo e output: sezione dedicata all'implementazione vera e propria del gestionale. Partendo da una presentazione delle tecnologie utilizzate si illustrano screenshots del progetto finito e si approfondiscono alcune funzionalità particolarmente rilevanti. 
    \item Testing: analisi e spunti sulla tecnica di verifica e validazione del software e presentazione di alcuni casi di test.   
    \item Conclusioni: sviluppi futuri e considerazioni personali.
\end{enumerate}

% \textcolor{blue}{Sommario è un breve riassunto del lavoro svolto dove si descrive l'obiettivo, l'oggetto della tesi, le 
% metodologie e le tecniche usate, i dati elaborati e la spiegazione delle conclusioni alle quali siete arrivati.} 

% Il sommario dell’elaborato consiste al massimo di 3 pagine e deve contenere le seguenti informazioni:
% \begin{itemize}
%   \item contesto e motivazioni 
%   \item breve riassunto del problema affrontato
%   \item tecniche utilizzate e/o sviluppate
%   \item risultati raggiunti, sottolineando il contributo personale del laureando/a
% \end{itemize}

\clearpage
\newpage



