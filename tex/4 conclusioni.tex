\chapter{Conclusioni}
\label{cha:conclusioni}
Il presente elaborato ha riportato ogni fase della progettazione del software gestionale sviluppato a supporto dell'area formazione dell'azienda Dream S.R.L. Si è collocato il progetto in relazione ai temi della gestione dei processi e dei Sistemi Informativi, illustrata l'analisi dei requisiti raccolti e la costruzione del database. Infine si è mostrata la parte di programmazione dell'applicativo e i test effettuati.\\
\newline
Si presentano ora gli scenari futuri del lavoro realizzato e, nella seconda sezione, si descrivono le considerazioni finali. 

\section{Sviluppi futuri}
\label{sec:futuro}
Il gestionale sviluppato è attualmente utilizzato dagli utenti dell'organizzazione, e disponibile online. In intesa con l'azienda ho stabilito un accordo per gestire e implementare anche le nuove funzionalità che saranno richieste in futuro, e per correggere gli errori che emergeranno durante l'utilizzo. Grazie alla suite usata sarà possibile intervenire in modo semplice sul codice per aggiungere moduli al progetto e controllare il rilascio delle varie versioni.\\
\newline
L'obiettivo è quello di continuare l'implementazione del gestionale in modo chiaro e strutturato, considerando anche che i nuovi requisiti da soddisfare saranno probabilmente relativi all'ottimizzazione di altri processi dell'area aziendale.     

\section{Considerazioni personali}
\label{sec:personali}
Il periodo di tirocinio è stato sicuramente positivo e istruttivo. Ho avuto modo non solo di assimilare nuove nozioni tecniche e di rafforzare quelle già in mio possesso, ma soprattutto di svolgere molte nuove attività. Sono consapevole che le esperienze fatte e le conoscenze apprese mi saranno utili nel proseguimento del mio percorso formativo, indipendentemente dalle scelte future che intraprenderò.\newline
Se da un lato parte delle fasi di analisi e di sviluppo sono state un approfondimento delle nozioni viste in alcuni corsi accademici seguiti come quello di Ingegneria del Software o di Basi di Dati, dall'altro ho avuto modo di acquisire nuove competenze trasversali legate alla comunicazione, al lavoro di gruppo e, in generale, alle capacità relazionali. \\
\newline
Realizzare un progetto a trecentosessanta gradi è stata un'esperienza molto stimolante e mi ha permesso di conoscere e capire i processi aziendali presenti, nonché lavorare in team su progetti condivisi, ma anche svolgere le attività in autonomia quando necessario.\\
\newline
Avendo vissuto ogni fase con positività, curiosità e volontà di apprendimento, mi ritengo soddisfatto del tirocinio e della stesura di questo elaborato nella sua totalità.\\
Tutto ciò, unitamente al mio percorso triennale di studi, mi ha permesso di crescere come persona rendendomi pronto ad affrontare nuove sfide.

  